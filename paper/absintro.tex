
\section*{Abstract}

Increasingly complex experiments and the benefit of sharing structured
but heterogeneous data demonstrate a clear need for a formal approach
to science.  Here, we propose a mathematical framework for
experimentation and analysis in physiology. First, we define a
structure for representing physiological observations. This structure
emphasises the critical role of time in physiology, but is flexible,
in that it can carry information of any type. Thus, we define an
ontology of physiological quantities that can describe a wide range of
observations. Second, we show that experiments themselves can be
composed from time-dependent quantities, and be expressed as purely
mathematical equations that can be manipulated algebraically. Our
framework is concise, allowing entire experiments to be defined
unambiguously in a few equations. To demonstrate the practicality and
versatility of our approach, we show the full equations for two
non-trivial implemented and analysed experiments describing visually
stimulated neuronal responses and dynamic clamp. The brevity of these
definitions illustrates the power of our approach, and we discuss its
implications for neuroinformatics-based research.

\pagebreak

\section*{Introduction}

Formalising scientific inference in mathematical frameworks removes
ambiguity and thus allows protocols to be formulated efficiently,
knowledge to be communicated transparently, and inferences to be
scrutinised \citep{Soldatova2006, Jaynes2003, Krantz1971}. Many
aspects of the scientific enterprise, including hypothesis testing,
estimation, and optimal parameter choice, are addressed rigorously in
\emph{statistics} and \emph{experimental design}. Nevertheless,
without knowing where observations come from, it is difficult to
ascertain whether they provide evidence for a given theory
\citep{Pool2002,MacKenzie-Graham2008,VanHorn2009}. Formalising the
experiments themselves is difficult because they produce heterogeneous
data \citep{Tukey1962}, and because experiments interact with the
physical world and therefore cannot be described purely by relations
between mathematical objects. Consequently, experiments are 
% invariably
almost always described in natural language and carried out manually or by
\emph{ad hoc} computer code. Attempts at formalisation have focused on the
execution of specific experiments \citep{Jenkins1989, Manduchi1990, King2004}
that seem difficult to generalise.

Whether they are carried out by humans or by automated equipment, many
experiments can be seen as \emph{programs} that manipulate and observe
the real world. This view suggests that experiment descriptions should
resemble programming languages. In creating a mathematical framework
for experiments, we take advantage of progress in embedding side
effects, such as input and output \citep{PeytonJones2002, Roy2004,
  Wadler1995} into purely equational programming languages, that is
languages that can only evaluate mathematical functions
\citep{Church1941}. These languages, unlike conventional programming
languages, retain an important characteristic of mathematics: a term
can freely be replaced by another term with identical meaning.
% HN 2010-09-30: Always "substitute for"
This property \citep[referential transparency;][]{Whitehead1927} enables
algebraic manipulation and reasoning about the programs
\citep{Bird1996}. This is important in the context of experiments,
because it means that an observation can always be defined with a
single equation.

Here, we formalise physiological experiments in a mathematical
framework. Our work is based on Functional Reactive Programming
\citep[FRP;][]{Elliott1997, Nilsson2002}, a concise and powerful
formulation of time-dependent reactive computer programs. We show that
there is substantial overlap between the concepts introduced by FRP
and physiological observations; consequently, physiological
experiments can be concisely defined in an FRP-like language. This
framework does not describe the physical components of biological
organisms. It has no concept of networks, cells or proteins. Instead
it describes the observation and calculation of the mathematical
objects that constitute physiological evidence (``observations'').

Our framework provides:

(i) An explicitly defined ontology of physiological observations. We
outline a flexible but concisely defined structure for physiological
quantities. This ontology can form the basis for repositories of
physiological data and meta-data. Physiological databases (unlike
those in bioinformatics or anatomy) have not found widespread adoption
\citep{Herz2008, Amari2002}, despite many attempts \citep{Katz2010,
  Teeters2008, Gardner2004, Jessop2010}. We suggest that the
flexibility of our ontology can remedy some of the structural
shortcomings of existing databases \citep{Gardner2005, Amari2002} and
thus facilitate data sharing \citep{Insel2003}.

(ii) A concise language for describing complex experiments and
analysis procedures in physiology. Thus experimental protocols can be
communicated unambiguously, highlighting differences between studies
and facilitating replication and meta-analysis. This language solves
the data provenance problem \citep{Pool2002} for experiments where the
primary pertubation is a complex but non-physical stimulus --- for
instance, a visual or electrical stimulation, rather than a chemical
substance.

(iii) A new approach to validating scientific inference
\citep{Editors2003, Editors2010, DeSchutter2010}. By inspecting an
experiment definition, automated decision procedures can verify
statements about experiments that indicate sound scientific practice,
such as consistent units of measurement \citep{Kennedy1997} and
correct error propagation \citep{Taylor1997}. The use of formal
languages can thus bring transparency to complex experiments and
analyses.

(iv) A practical tool that is powerful and generalises to complex and
multi-modal experiments. We have implemented our framework as a new
programming language and used it for non-trivial neurophysiological
experiments and data analyses.

Here, we first describe the theory of \emph{simple types}
\citep{Pierce2002} and define three types that can represent
physiological evidence. We then present a new formal and
machine-executable language, the \emph{calculus of physiological
  evidence}, for defining observations and transformations of such
evidence. Finally, we show that two very different experiments from
neurophysiology can be formally defined, run and analysed in our
calculus. In the first example, we measure \emph{in vivo} spike train
responses to visual stimulation in locusts. In the second example, we
examine the impact of an active potassium conductance on synaptic
integration using the dynamic clamp technique. These protocols are
defined unambiguously using only a handful of equations in our
language.

% HN 2010-09-30: Introduce CPE as an abbreviation? Would (possibly)
% address a concern of Angus's.
