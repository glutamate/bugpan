\section*{Abstract}
  Experiments can be complex and produce large volumes of
  heterogeneous data, which makes their execution, analysis,
  independent replication and meta-analysis difficult. We propose a
  mathematical model for experimentation and analysis in physiology
  that addresses these problems. We show that experiments can be
  composed from time-dependent quantities, and be expressed as purely
  mathematical equations. Our structure for representing physiological
  observations can carry information of any type and therefore provides
  a precise ontology for a wide range of observations. Our framework
  is concise, allowing entire experiments to be defined unambiguously
  in a few equations. To demonstrate that our approach can be
  implemented, we show the full equations we have used to run and
  analyse two non-trivial experiments describing visually stimulated
  neuronal responses and dynamic clamp of vertebrate neurons. Our
  ideas could provide a theoretical basis for developing new standards
  of data acquisition, analysis and communication in neurophysiology.

\pagebreak

\section*{Introduction}

Reproducibility and transparency are cornerstones of the scientific
method. As scientific experiments and analysis become increasingly
complex, reliant on computer code and produce larger volumes of data,
the feasibility of independent verification and replication has -- in
practice -- been undermined. Primary data sharing is now standard in
applications of high-throughput biology, but not in the many fields
that produce heterogeneous observations \citep{Gardner2005,
  Tukey1962}. Even when raw data and code used for experiments and
analysis are fully disclosed, only a minority of findings can be
reproduced without discrepancies \citep{Ioannidis2008,Baggerly2009,
  McCullough2007}. It is often not realistic to verify that computer
code used in analyses correctly implements an intended mathematical
algorithm, yet errors can undermine a large body of work
\citep{Chang2006}. The combination of bias, human error, and unverified
software, has led to the suggestion that many published research
findings are flawed \citep{Ioannidis2005, Merali2010}.

Some of these problems may be mitigated by developing explicit models
of experimentation, evidence representation and analysis. A good model
should simultaneously: (i) introduce a system of categorisation
directly relevant to the scientific field, such that scientists can
define experiments and reason about observations in familiar terms,
(ii) be machine executable, therefore unambiguous and practically
useful, and (iii) consist exclusively of terms that directly
correspond to mathematical entities, enabling algebraic reasoning
about and manipulation of procedures.  Experiments are difficult to
formalise in terms of relations between mathematical objects because
they produce heterogeneous data \citep{Tukey1962}, and because they
interact with the physical world. In creating a mathematical framework
for experiments, we take advantage of progress in embedding input and
output \citep{PeytonJones2002, Wadler1995} into programming languages
where the only mechanism of computation is the evaluation of
mathematical functions \citep{Church1941}.

Here, we define a formal framework for physiology
that satisfies the above criteria. We show that there
is a large conceptual overlap between physiological experimentation
and Functional Reactive Programming (FRP; \citep{Elliott1997,
  Nilsson2002}), a concise and purely functional formulation of
time-dependent reactive computer programs. Consequently, physiological
experiments can be concisely defined in the vocabulary of
\emph{signals} and \emph{events} introduced by FRP. Such a language
does not describe the physical components of biological organisms; it
has no concept of networks, cells or proteins. Instead it describes
the observation and calculation of the mathematical objects that
constitute physiological evidence (``observations'').

Our framework provides:

(i) An explicitly defined ontology of physiological
observations. Physiological databases have not been widely
adopted\citep{Herz2008, Amari2002}, unlike in bioinformatics or
anatomy, despite many candidates \citep{Jessop2010, Teeters2008,
  Frishkoff2009, Katz2010}.  We suggest that a flexible, concise and
simple structure for physiological quantities can remedy some of the
shortcomings \citep{Gardner2005, Amari2002} of existing databases and
thus facilitate the sharing of data and meta-data \citep{Insel2003}.

(ii) A concise language for describing complex experiments and
analysis procedures in physiology using only mathematical
equations. Experimental protocols can be communicated unambiguously,
highlighting differences between studies and facilitating replication
and meta-analysis. The provenance \citep{Pool2002,MacKenzie-Graham2008,
  VanHorn2009} of \emph{any} observation can be extracted as a single
equation that includes post-acquisition processing and censoring. In
addition, analysis procedures in languages with a clear mathematical
denotation are verifiable since their implementation closely
follows their specification\citep{Bird1996}.

(iii) The theoretical basis for new tools that are practical, powerful
and generalise to complex and multi-modal experiments. To demonstrate
this, we have implemented our framework as a new programming language
and used it for non-trivial neurophysiological experiments and data
analyses. A strength of our approach is that its individual elements
could, alternatively, be adopted separately or in different ways to
suit the different demands.
