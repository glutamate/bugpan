
\section*{Abstract}

Experiments can be complex and produce large volumes of heterogeneous
data, which makes their execution, analysis, independent replication
and meta-analysis difficult. Here, we propose a mathematical model for
experimentation and analysis in physiology that addresses these
problems. First, we define a structure for representing physiological
observations. This structure emphasises the critical role of time in
physiology, but is flexible, in that it can carry information of any
type. Thus, we define an ontology of physiological quantities that can
describe a wide range of observations. Second, we show that
experiments themselves can be composed from time-dependent quantities,
and be expressed as purely mathematical equations that can be
manipulated algebraically. Our framework is concise, allowing entire
experiments to be defined unambiguously in a few equations. To
demonstrate the practicality and versatility of our approach, we show
the full equations for two non-trivial implemented and analysed
experiments describing visually stimulated neuronal responses and
dynamic clamp of vertebrate neurons. The brevity of these definitions
illustrates the power of our approach, and we discuss its implications
for neuroinformatics-based research.

\pagebreak

\section*{Introduction}

Reproducibility is a cornerstone of the scientific method. As
scientific experiments and analysis are become increasingly complex,
reliant on computer code and produce larger volumes of data, the
feasibility of independent verification and replication has -- in
practice -- been undermined. Full disclosure of raw data and code for
experiments and analysis does not guarantee reproducibility.  For
instance, in only 8 out of 16 studies subject to this requirement and
published in \emph{Nature Genetics} could the findings be replicated
at all, and in only 2 of those, without discrepancies
\cite{Ioannidis2008} (see also \cite{Baggerly2009,
  McCullough2007}). It is often not realistic to verify that computer
code used in analyses correctly implements an intended mathematical
algorithm, yet errors can undermine the conclusions of a
large body of work \cite{Chang2006}.  In addition, primary data
sharing is standard in areas of high-throughout biology, but
not in fields that produce heterogeneous data \cite{Gardner2005},
further undermining independent validation.  Indeed, the combination
of bias, human error, and unverified software, has led to the
suggestion that many --- perhaps most --- published research findings
contain serious errors \cite{Ioannidis2005, Merali2010}.

The problems outlined above suggest that it is worth investigating a
unified and transparent formalism for communication, validation, and
automation of observations, experiments and analyses. We argue that
some of the above problems can be mitigated by a formalism that
% \emph{at once}: 
\emph{simultaneously}: (i) introduces a system of categorisation
directly relevant to the scientific field, such that scientists can
define experiments and reason about observations in familiar terms,
(ii) is machine executable, therefore unambiguous and practically
useful, and (iii) is composed of terms that directly correspond to
mathematical entities, unlike conventional programming languages,
enabling algebraic manipulation and certification of the experiments
and analysis procedures.  Experiments are difficult to formalise in
terms of relations between mathematical objects because they
produce heterogeneous data \cite{Tukey1962}, and because they
interact with the physical world. In creating a mathematical framework
for experiments, we take advantage of progress in embedding side
effects, such as input and output \cite{PeytonJones2002, Roy2004,
  Wadler1995}, into equational programming languages; that is, languages
that can only evaluate mathematical functions \cite{Church1941}.

Here, we define a formal framework for physiology
that satisfies the above criteria. We show that there
is a large conceptual overlap between physiological experimentation
and Functional Reactive Programming (FRP\cite{Elliott1997,
  Nilsson2002}), a concise and purely functional formulation of
time-dependent reactive computer programs. Consequently, physiological
experiments can be concisely defined in the vocabulary of
\emph{signals} and \emph{events} introduced by FRP. Such a language
does not describe the physical components of biological organisms. It
has no concept of networks, cells or proteins. Instead it describes
the observation and calculation of the mathematical objects that
constitute physiological evidence (``observations'').

Our framework provides:

(i) An explicitly defined ontology of physiological observations. We
outline a flexible but concisely defined structure for physiological
quantities.  Physiological databases have not been widely adopted,
unlike in bioinformatics or anatomy \cite{Herz2008,
  Amari2002}. Existing candidates support only a few different kinds
of data \cite{Jessop2010}, are unstructured \cite{Teeters2008},
based on a large number of concepts \cite{Frishkoff2009}, or represent
facts without their supporting evidence \cite{Katz2010}.
%despite many attempts \cite{Katz2010, , Gardner2004, Jessop2010}.  
We suggest that a flexible and simple ontology can remedy some of the
structural shortcomings\cite{Gardner2005, Amari2002} of existing
databases and thus facilitate the sharing of data and meta-data
\cite{Insel2003}.

(ii) A concise language for describing complex experiments and
analysis procedures in physiology. Experimental protocols can be
communicated unambiguously, highlighting differences between studies
and facilitating replication and meta-analysis. Our language thus
describes the the data provenance
\cite{Pool2002,MacKenzie-Graham2008, VanHorn2009} for experiments
with that can be fully executed by a computer. By construction,
\emph{any} observation can be defined in a single equation that must
include post-acquisition processing and censure.

(iii) A new approach to validating scientific inference
\cite{Editors2003, Editors2010}. First, a concise language with a
clear mathematical denotation can be verified manually. Second, by
inspecting an experiment definition, \emph{automated} decision procedures
could verify certain statements about experiments that indicate sound
scientific practice, such as consistent units of measurement
\cite{Kennedy1997} and correct error propagation \cite{Taylor1997}.

(iv) A practical tool that is powerful and generalises to complex and
multi-modal experiments. We have implemented our framework as a new
programming language and used it for non-trivial neurophysiological
experiments and data analyses. This language can concisely express and
run experiment with stimuli defined by, for instance, differential
equations or complex animations.

We first describe the theory of \emph{simple types}
\cite{Pierce2002, Hindley2008} and define three types that can represent a wide
range of physiological evidence. We then present a new formal and
machine-executable language, the \emph{calculus of physiological
  evidence} (CoPE), for defining observations and transformations of
such evidence. We show that two very different experiments from
neurophysiology can be formally defined, run and analysed in our
calculus. In the first example, we elicit and measure \emph{in vivo}
spike train responses to visual stimulation in locusts. In the second
example, we use the dynamic clamp technique to examine the impact of
an active potassium conductance on synaptic integration in zebrafish
neurons recorded \emph{in vivo}. These protocols are defined
unambiguously using only a handful of equations in CoPE.

\begin{comment}

Then, before the concluding
discussion, we sketch how the FRP-like language described here,
centred around descriptions in the time domain, could be generalised
for other domains, like frequency or space.



\section*{What Henrik Wrote}

Independent verification and replication of results is one of the cornerstones
of the scientific process. A prerequisite for this cross-checking is that
published results are accompanied by a sufficiently detailed description of
how they were obtained along with access to any essential raw data. However,
as software increasingly is becoming an integral part of the scientific
process, the feasibility of independent verification and replication has
gradually been undermined. This was highlighted by last year's controversy
over climate-science results brought about by leaked e-mails from the
University of East Anglia in Norwich, UK. These e-mails called into question
the quality of some of the developed, non-public, software that had been used,
leading the official inquiry to call for scientific code to be published
as a matter of course.

A couple of recent articles in Nature \cite{Merali2010, Barnes2010} discuss this incident 
and others, including one case where a structural-biology group had to
retract five published papers after an error had been discovered in a program
that had been used to prepare the data. A number of reasons are given for
this state of affairs, including that scientists are inadequately prepared for
increasingly prevalent and complex software development tasks, resulting
in poor quality code that they then, to make matters worse, are reluctant to
disclose.

However, it is also pointed out that disclosure of software alone is unlikely
to solve the problem. Verifying conventional code is both very difficult and
very time-consuming, and thus it would be too much to ask for this to happen
as part of the peer review process. Moreover, given the rapid development of
hardware and software environments, archived code can quickly become next to
impossible to run. As an example, the \textit{Journal of Money, Credit, and
Banking} have required archiving of software and data associated with
published papers for over a decade. However, when the effectiveness of this
requirement was evaluated, it turned out that it was only possible to
independently replicate the results for fewer than 10 \% of the 150 papers
that had been subject to the archiving policy over a 10-year period.

Now, these problems are of course complex and multi-faceted, so there is not
going to be any one way to address them all. However, as a part of a solution,
the discussion above suggests that it would be valuable to investigate new
approaches to how scientific experiments, observation, and analysis are
described. To that end, we propose in the following an approach that we argue
mitigates some of the outlined problems. We do this in the setting of
physiology, but the basic ideas should have much wider applicability.

Central to our approach is a mathematically defined, high-level, language for
describing physiological experiments and subsequent observation and analysis
in a way that is precise, concise, and phrased in terms that are immediately
relevant to the problem domain and thus familiar to the scientists themselves.
We believe this approach offers a number of advantages pertinent to the
preceding discussion:
\begin{itemize}
\item
    Because the language is based on vocabulary that is immediately
    scientifically relevant, and because it is precise and concise, it can
    serve as an effective way to communicate science, thus making verification,
    for example, as part of the peer review process, and independent
    replication feasible.
\item
    Because descriptions are precise, verification of some aspects
    can even be mechanised.
\item
    Because descriptions are precise, they can be mechanically translated
    into executable form, thus in many cases obviating the need for
    developing software in conventional languages.
\item
    Because of familiarity with the vocabulary, it should be both
    easier and less error prone for scientists to describe experiments,
    observation, and analysis in this way compared with writing code in
    conventional programming languages.    
\item
    Because of the conciseness, our approach is also significantly
    less laborious compared with coding in conventional languages.
\item
    Because the language has a clear, mathematically defined meaning,
    a description stands on its own and the value of archived descriptions
    do not deteriorate over time as hardware and software environments
    evolve.
\end{itemize}

\section*{Old Introduction}

Formalising scientific inference in mathematical frameworks removes
ambiguity and thus allows protocols to be formulated efficiently,
knowledge to be communicated transparently, and inferences to be
scrutinised \cite{Soldatova2006, Jaynes2003, Krantz1971}. Many
aspects of the scientific enterprise, including hypothesis testing,
estimation, and optimal parameter choice, are addressed rigorously in
\emph{statistics} and \emph{experimental design}. Nevertheless,
without knowing where observations come from, it is difficult to
ascertain whether they provide evidence for a given theory
\cite{Pool2002,MacKenzie-Graham2008,VanHorn2009}. Formalising the
experiments themselves is difficult because they produce heterogeneous
data \cite{Tukey1962}, and because experiments interact with the
physical world and therefore cannot be described purely by relations
between mathematical objects. Without a formal language, it is
difficult to reason precisely. Consequently, experiments are often
communicated in natural language and carried out by \emph{ad hoc}
computer code that is difficult to validate \cite{Baggerly2009, Merali2010}. 
%Attempts at formalisation have focused on
%the execution of specific experiments \cite{Jenkins1989,
%  Manduchi1990, King2004} that seem difficult to generalise.

Whether they are carried out by humans or by automated equipment, many
experiments can be seen as \emph{programs} that manipulate and observe
the real world. This view suggests that experiment descriptions should
resemble programming languages. In creating a mathematical framework
for experiments, we take advantage of progress in embedding side
effects, such as input and output \cite{PeytonJones2002, Roy2004,
  Wadler1995} into purely equational programming languages, that is
languages that can only evaluate mathematical functions
\cite{Church1941}. These languages, unlike conventional programming
languages, retain an important characteristic of mathematics: a term
can freely be replaced by another term with identical meaning.
% HN 2010-09-30: Always "substitute for"
This property \cite[referential transparency;][]{Whitehead1927} enables
algebraic manipulation and reasoning about the programs
\cite{Bird1996}. This is important in the context of experiments,
because it means that an observation can always be defined with a
single equation.

Here, we show that there is a large conceptual overlap between
physiological experimentation and Functional Reactive Programming
\cite[FRP;][]{Elliott1997, Nilsson2002}, a concise formulation of
time-dependent reactive computer programs. Consequently, physiological
experiments can be concisely defined in an FRP-like language. Such a
language does not describe the physical components of biological
organisms. It has no concept of networks, cells or proteins. Instead
it describes the observation and calculation of the mathematical
objects that constitute physiological evidence (``observations'').

This framework provides:

(i) An explicitly defined ontology of physiological observations. We
outline a flexible but concisely defined structure for physiological
quantities.  Unlike in bioinformatics or anatomy, physiological
databases have not found widespread adoption \cite{Herz2008,
  Amari2002}. Existing candidates either support only a few different
kinds of data \cite{Jessop2010}, are unstructured \cite{Teeters2008}
or based on a large number of concepts \cite{Frishkoff2009}, or
represent facts without their supporting evidence \cite{Katz2010}.
%despite many attempts \cite{Katz2010, , Gardner2004, Jessop2010}.  
We suggest that a flexible and simple ontology
can remedy some of the structural shortcomings of existing databases
\cite{Gardner2005, Amari2002} and thus facilitate data and meta-data
sharing \cite{Insel2003}.

(ii) A concise language for describing complex experiments and
analysis procedures in physiology. Thus experimental protocols can be
communicated unambiguously, highlighting differences between studies
and facilitating replication and meta-analysis. This language solves
the data provenance problem \cite{Pool2002} for experiments where the
primary pertubation is a complex but non-physical stimulus --- for
instance, a visual or electrical stimulation, rather than a chemical
substance.

(iii) A new approach to validating scientific inference
\cite{Editors2003, Editors2010, DeSchutter2010}. By inspecting an
experiment definition, automated decision procedures could verify
statements about experiments that indicate sound scientific practice,
such as consistent units of measurement \cite{Kennedy1997} and
correct error propagation \cite{Taylor1997}. The use of formal
languages can thus bring transparency to complex experiments and
analyses.

(iv) A practical tool that is powerful and generalises to complex and
multi-modal experiments. We have implemented our framework as a new
programming language and used it for non-trivial neurophysiological
experiments and data analyses. This language can concisely express and
run experiments defined by, for instance, differential equations or
visual stimuli.

Here, we first describe the theory of \emph{simple types}
\cite{Pierce2002} and define three types that can represent
physiological evidence. We then present a new formal and
machine-executable language, the \emph{calculus of physiological
evidence}, for defining observations and transformations of such
evidence. We proceed to show that two very different experiments from
neurophysiology can be formally defined, run and analysed in our
calculus. In the first example, we measure \emph{in vivo} spike train
responses to visual stimulation in locusts. In the second example, we
examine the impact of an active potassium conductance on synaptic
integration using the dynamic clamp technique. These protocols are
defined unambiguously using only a handful of equations in our
language. Then, before the concluding discussion, we sketch how the FRP-like
language described here, centred around descriptions in the time domain,
could be generalised for other domains, like frequency or space.


% HN 2010-09-30: Introduce CPE as an abbreviation? Would (possibly)
% address a concern of Angus's.
\end{comment}
