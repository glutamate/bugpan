\section*{Discussion}

We present a new approach to performing and communicating experimental
science.  Our use of typed, functional and reactive programming
overcomes at least two long-standing issues in bioinformatics: the
need for a flexible ontology to share heterogeneous data from
physiological experiments \citep{Amari2002}, a language for describing
experiments and data provenance unambiguously \citep{Pool2002,
  Murray-Rust2002}.

The types we have presented form a linguistic framework and an
ontology for physiology. Thanks to the flexibility of parametric
polymorphism, our ontology can form the basis for the interchange of
physiological data and meta-data without imposing unnecessary
constraints on what can be shared. The ontology is non-hierarchical
and would be difficult to formulate in the various existing semantic
web ontology frameworks (Web Ontology Language, \citep{owlref}, or
Resource Description Framework), which lack parametric polymorphism
and functional abstraction. Nevertheless, by specifying the categories
of mathematical objects that constitute evidence, it is an ontology in
the classical sense of cataloguing the categories within a specific
domain, and providing a vocabulary for that domain. We emphasise again
that it is an ontology of \emph{evidence}, not of the biological
entities that give rise to this evidence. It is unusual as an ontology
for scientific knowledge in being embedded in a \emph{computational}
framework, such that it can describe not only mathematical objects but
also their transformations and observations. Recent work on meta-data
representation \citep{Bower2009} has focused on delineating the
information needed to repeat an experiment \citep{Taylor2007,
  Gibson2008}, but in practice it is often not clear \emph{a priori}
what aspects of an experiment could influence its outcome. With CoPE,
our main goal was to describe unambiguously machine-executable aspects
of the meta-data of an experiment. Other information can nevertheless
be added on an \emph{ad hoc} basis, and it can exploit the temporal
contexts provided by CoPE, i.e. it can exist as signals, events or
durations. Thus, we make no fundamental distinction between the
representation of data and meta-data. All relevant information about
an experiment is indexed by time and thus linked by overlap on a
common time scale.

Parametric polymorphism and first-class functions are generally
associated with research-oriented languages such as Haskell and ML
rather than mainstream programming languages such as C++ or Java. It
is likely that much of CoPE could be implemented in C++, where
template metaprogramming implements a static form of parametric
polymorphism and template functors can be used to represent
functions. These must all be resolved at compile-time, however, so it
is difficult to use dynamically calculated or arbitrarily complex
functions. The importance of true first-class functions and parametric
polymorphism is becoming increasingly well recognized, and these
features are now being implemented in the mainstream programming
languages C++, C\#, and Java. We therefore expect that it will
soon be possible to implement the formalism we are proposing in a wide
range of programming languages.

% HN 2010-11-24: Instead of
%
% Our mathematical definitions are less ambiguous than definitions
% written in plain English, 
% 
% to make it consistent with "unambiguous" above, I propose
%
Our mathematical definitions are unambiguous and concise, unlike
typical definitions written in natural language, and are more powerful
than those specified by graphical user interfaces or in formal
languages that lack a facility for defining abstractions. Our
framework is not only a theoretical formalism, but we also demonstrate
that it can be implemented as a very practical tool. This tool
consists of a collection of computer programs for executing
experiments and analyses, and carrying out data
management. Controlling experimentation and analysis with programs
that share data formats can be highly efficient, and eliminates many
sources of human error. Existing tools use differential equations to
define dynamic clamp experiments (Model Reference Current Injection
(MRCI); \citep{Raikov2004}) or simulations (X-Windows Phase Plane
(XPP); \citep{Ermentrout1987}). Here, we show that a general
(polymorphic) defintion of signal and events, embedded in the lambda
calculus, can define a much larger range of experiments, and the
evidence that they produce. Our experiment definitions have the
further advantage that they \emph{compose}; that is, more complex
experiments can be formulated by joining together elementary building
blocks.

Our full approach is particularly relevant to the execution of very
complex and multi-modal experiments, which may need to be dynamically
reconfigured based on previous observations, or to disambiguate
difficult judgements about evidence \citep{Kriegeskorte2009}. Even if
used separately, however, individual aspects of CoPE can make distinct
contributions to scientific methodology. For instance, our ontology
for physiological evidence can be used within more conventional
programming languages or web applications that facilitate data
sharing. In a similar way, the capabilities of CoPE for executing and
analysing experiments could provide a robust core for innovative
graphical user interfaces. We expect the formalism presented here to
be applicable outside neurophysiology. Purely temporal information
from other fields could be represented using signals, events and
durations, or other kinds of temporal context formalised in type
theory. In addition, the concepts of signals, events and durations
could be generalised to allow not just temporal but also spatial or
spatiotemporal contexts to be associated with specific values. Such a
generalisation is necessary for CoPE to accommodate data from the
wider neuroscience community, including functional neuroanatomy and
microscopy, and other scientific diciplines that observe and
manipulate spatiotemporal data.

We have argued that observational data, experimental protocols and
analyses formulated in CoPE (or similar frameworks) are less ambiguous
and more transparent than those described using many contemporary
formulations. The structure of CoPE also presents additional
opportunities for mechanically excluding some types of procedural
errors in drawing inferences from experiments. For instance, CoPE
could incorporate an extension to simple type theory
\citep{Kennedy1997} that adds not only a consistency check for
dimensional units but also powerful aspects of dimensional analysis,
such as Buckingham's $\pi$-theorem. Furthermore, in our formulation,
experimentally observed values exist as mathematical objects within a
computational framework that can be used to define probability
distributions. Hierarchical probabilistic notation \citep{Gelman2006}
permits the construction of flexible statistical models for the
directly observed data. Integrating our formalism for experiments and
heterogeneous observations with statistical programming languages,
such as WinBUGS \citep{Gilks1994}, AutoBayes \citep{AutoBayes} or
Mlwin \citep{mlwin}, that estimate model parameters using Bayesian or
maximum likelihood methods could be used to quantify different aspects
of uncertainty in the measurements, taking into account all available
information.

\section*{Experimental Procedures}

\subsubsection*{Language implementation}

We have used two different implementation strategies for reasons of
rapid development and execution efficiency. For purposes of
experimentation and simulation, we have implemented a prototype
compiler that can execute some programs that contain signals and
events defined by mutual recursion, as is necessary for the
experiments in this paper. The program is transformed by the compiler
into a normal form that is translated to an imperative program which
iteratively updates variables corresponding to signal values, with a
time step that is set explicitly. The program is divided into a series
of stages, where each stage consists of the signals and events defined
by mutual recursion, subject to the constraints of input/output
sources and sinks. This ensures that signal expressions can reference
values of other signals at arbitrary time points (possibly in the
future) as long as referenced signals are computed in an earlier
stage.

To calculate a new value from existing observations after data
acquisition, we have implemented the calculus of physiological
evidence as a domain-specific language embedded in the purely functional
programming language Haskell.

For hard real-time dynamic-clamp experiments, we have built a compiler
back-end targeting the LXRT (user-space) interface to the RTAI (Real-time
application interface; \url{http://rtai.org}) extensions of the Linux
kernel, and the Comedi (\url{http://comedi.org}) interface to data
acquisition hardware. Geometric shapes were rendered using OpenGL
(\url{http://opengl.org}).

All code used for experiments, data analysis and generating figures is
available at \url{http://github.com/glutamate/bugpan} under the GNU General
Public License (GPL).

\subsubsection*{Locust experiments}

Locusts were maintained at 1,600 m$^{-3}$ in 50$\times$50$\times$50 cm cages under a
standard light and temperature regime of 12h light at 36$^{\circ}$C : 12h dark
at 25$^{\circ}$C. They were fed ad lib with fresh wheat seedlings and bran
flakes. Recordings from locust DCMD neurons were performed as
previously described \citep{Matheson2004}. Briefly, locusts were fixed
in plasticine with the ventral side upwards. The head was fixed with
wax
% at a $90^{\circ}$ angle
and the connectives were exposed through an
incision in the soft tissue of the neck. A pair of silver wire hook
electrodes were placed underneath a connective and the electrodes
and connective enclosed in petroleum jelly. The electrode signal was
amplified 1000x and bandpass filtered 50--5000 Hz, before
analog-to-digital conversion at 18 bits and 20 kHz with a National
Instruments PCI-6281 board. The locust was placed in front of a 22''
CRT monitor running with a vertical refresh rate of 160 Hz. All
aspects of the visual stimulus displayed on this monitor and of
the analog-to-digital conversion performed by the National Instruments
board were controlled by programs written in
% the calculus of physiological evidence
CoPE running on a single computer.

The full code for the locust experiment is given in Listing 1
(Supplementary Materials).

\subsubsection*{Zebrafish experiments}

Zebrafish were maintained according to established procedures
\citep{Westerfield1994} in approved tank facilities, in compliance
with the Animals (Scientific Procedures) Act 1986 and according to
University of Leicester guidelines. Intracellular patch-clamp
recordings from motor neurons in the spinal cord of a 2-day old
zebrafish embryo were performed as previously described
\citep{McDearmid2006}. We used a National Instruments PCI-6281 board
to record the output from a BioLogic patch-clamp amplifier in
current-clamp mode, filtered at 3kHz and digitised at 10 kHz, with the
output current calculated at the same rate by programs written in
% the calculus of physiological evidence
CoPE targeted to the RTAI backend (see
above). The measured jitter for updating the output voltage was 6
$\mu$s and was similar to that measured with the RTAI latency test
tool for the experiment control computer.

The full code for the Zebrafish experiments is given in Listing 2
(Supplementary Materials).

\subsubsection*{Author Contributions}  

T.N. designed and implemented CoPE, carried out the experiments and
data analyses, and wrote the draft of the paper. H.N. contributed to
the language design, helped clarify the semantics, and wrote several
sections of the manuscript. T.M. contributed to the design of the
experiments and the data analysis, and made extensive comments on
drafts of the manuscript. All authors obtained grant funding to
support this project as described in the acknowledgements.

\subsubsection*{Acknowledgements}  

We would like to thank Jonathan McDearmid for help with the Zebrafish
recordings and Angus Silver, Guy Billings, Antonia Hamilton, Nick
Hartell and Rodrigo Quian Quiroga for critical comments on the
manuscript. This work was funded by a Human Frontier Science Project
fellowship to T.N., a Biotechnology and Biological Sciences Research
Council grant to T.M. and T.N., and Engineering and Physical Sciences Research
Council grants to H.N.



