\subsection*{Generalising Signals}
\label{sec-gensig}

% HN 2010-10-17: Used to be:
% \subsection*{Limitations and extensions}

The framework for physiological observations and experiments presented
in the previous sections does not include the ability to record or
manipulate spatial data from e.g. calcium imaging or animal behaviour studies,
or frequency domain representations of periodically repeating
observations. We have focused on electrophysiology because it has a
more limited scope and because FRP exclusively deals with temporal and
not spatial contexts. Nevertheless, generalising CoPE to include
spatial and frequency-domain data is in principle straightforward;
we outline this extension here. This more general theory potentially extends
our framework for observations, experiments and analyses to many areas outside
physiology.

Signals are generalised in the index type such that values can
be indexed by \emph{any} data type; for example, time for time-varying
quantities, frequency for spectra or periodograms, or vectors of
integers or real numbers for images:
\begin{tabbing}
\qquad\=\hspace{\lwidth}\=\hspace{\cwidth}\=\+\kill
$\Conid{Signal}\;\alpha\;\beta\mathrel{=}\alpha \to \beta$
\end{tabbing}
We merely require functions that define how values (of any type but of
known size) can be organised in memory for the given index type.
% HN 2010-11-10: I think the following is a little too much detail,
% I'm not sure that what it seems to suggests is central, and it
% is a bit inconsistent in that the actual transforms mentioned
% only concernt time-frequency. I've added a brief note on transfoms later
% instead.
%
% We define an operation from an index type to its \emph{dual}. For
% instance, frequency is the dual of time, and spatial frequency the
% dual of spatial length. We then finally define operations to
% calculate signals indexed by a type from signals indexed by the
% dual type, to represent transforms between temporal and frequency
% domains.
Events are also generalised to lists of pairs of index and value types: 
\begin{tabbing}
\qquad\=\hspace{\lwidth}\=\hspace{\cwidth}\=\+\kill
${\Conid{Event}\;\alpha\;\beta\mathrel{=}[\mskip1.5mu \alpha\;\!\!\times\!\!\;\beta\mskip1.5mu]}$
\end{tabbing}
This allows nesting of spatial and temporal contexts such that movies
(time-signals of images of, e.g., colour) or time-varying quantities
recorded at a particular location \citep[e.g., spot calcium
measurements as space-events of time-signals of
concentration;][]{DiGregorio1999} could be represented without
introducing new concepts. Moreover, where needed, standard transforms
can be introduced to mediate between different kinds of signals; for example,
Fourier transforms to mediate between time and frequency domain signals.

These spatial types and their transformations could also define visual
stimuli, providing a way to replace the arbitrary and limited
geometric primitives demonstrated in Example 1. This extension of CoPE
will require a more powerful type system than the one we have described.
previously. We have already implemented the central features in the purely
functional language Haskell using extensions that enable type-level
programming \citep{Kiselyov2010}.

