\section*{Generalised Signals}
\label{sec-gensig}

% HN 2010-10-17: Used to be:
% \subsection*{Limitations and extensions}

The framework for physiological observations and experiments presented
in the previous sections does not include the ability to record or
manipulate spatial data from e.g. calcium imaging or animal behaviour,
or frequency domain representation of periodically repeating
observations. We have focused on electrophysiology because it is a
more limited scope and because FRP exclusively deals with temporal and
not spatial contexts. Nevertheless, generalising CoPE to include
spatial and frequency-domain data is straightforward; we outline this
extension here. This more general theory potentially extends our
framework for observations, experiments and analyses to outside
physiology.

Signals are generalised in the index type, such that values can
be indexed by \emph{any} data type, such as time for time-varying
quantities, frequency for spectra or periodograms, or vectors of
integers or real numbers for images.
\begin{tabbing}
\qquad\=\hspace{\lwidth}\=\hspace{\cwidth}\=\+\kill
$\Conid{Signal}\;\alpha\;\beta\mathrel{=}\alpha \to \beta$
\end{tabbing}
We merely require functions that define how values (of any type but of
known size) can be organised in memory for the given index type. We
define an operation from an index type to its \emph{inverse}. For
instance, frequency is the inverse of time, and spatial frequency the
inverse of spatial length. We then finally define operations to
calculate signals indexed by a type from signals indexed by the
inverse type, to represent transforms between temporal and frequency
domains. Events are also generalised to lists of pairs of index and
value types. 
\begin{tabbing}
\qquad\=\hspace{\lwidth}\=\hspace{\cwidth}\=\+\kill
${\Conid{Event}\;\alpha\;\beta\mathrel{=}[\mskip1.5mu \alpha\;\!\!\times\!\!\;\beta\mskip1.5mu]}$
\end{tabbing}
This allow nesting of spatial and temporal contexts such that movies
(time-signals of images of, say, colour) or time-varying quantities
recorded at a particular location \citep[for instance spot calcium
measurements as space-events of time-signals of
concentration;][]{DiGregorio1999} could be represented without
introducing new concepts.  

These spatial types and their transformations could also define visual
stimuli, providing a way to replace the arbitrary and limited
geometric primitives in Example 1. This extension of CoPE relies on a
more powerful type system than the one we have described
previously. We have implemented the central features in the purely
functional language Haskell using extensions that enable type-level
programming \citep{Kiselyov2010}.

Generalising durations is more difficult, because there are many ways
of representing a volume in spaces with two or more dimensions:
implicit (ref), parametric (ref), or bounded by polygons. We do not
commit to any of these representations here.

