\documentclass[11pt]{article}
\usepackage[a4paper, top=2.5cm, bottom=2.5cm, left=2.5cm, right=2.5cm]{geometry}
\begin{document}
\begin{flushright}
Tom Matheson

Department of Biology

University of Leicester

University Road

Leicester LE1 7RH

UK
\end{flushright}
\vskip2ex
Dear Sir or Madam,
\vskip2ex
\noindent Please find enclosed our paper titled ``A formal
mathematical framework for physiological observations, experiments and
analyses'' for submission to \emph{Nature Methods}. Our work directly
addresses recent concerns about the reproducibility and correctness of
many published findings (e.g. Ioannidis, 2005, PLoS Medicine
2(8)). Here, we present a mathematical model of physiological
experimentation using concepts from programming language theory and
type theory. We show that this approach gives solutions to several
outstanding problems in neuroinformatics:

\begin{itemize}

\item \emph{Ontology}: the organisation of data sharing requires a
  structure for representing a wide range of observations and
  contextual information. We propose a simple but flexible
  categorisation of physiological evidence, phrased in terms that are
  directly relavant to physiology.

\item \emph{Experiment definition}: we show that experiments defined
  in mathematical equations can be manipulated algebraically and
  composed from smaller components, unlikes those defined in
  conventional programming languages. These definitions are also
  machine executable, therefore practically useful and unambiguous.

\item \emph{Data provenance}: without knowning where observed data
  comes from, it can be difficult to relate it to a specific
  theory. In our framework, \emph{any} observation can be defined in a
  \emph{single} equation that includes post-acqusition processing and
  censure.

\end{itemize}

We have validated this framework with two non-trivial experiments in
neurophysiology; one to measure visually evoked \emph{in vivo} spike
trains, and a dynamic clamp study of spinal neurons. This work has
serious implications for the the structure of data sharing and
publishing: our definitions \emph{cannot} be translated to ontology
languages used to define the semantic web. Although in this paper we
have focused on physiology, we outline a generalized theory and
discuss its interaction with statistics.

We suggest the following reviewers for our paper:

\begin{itemize}

\item Angus suggests: Boris Barbour, Hugh Robinson, Fabrizio Gabbiani
  Christoph Koch

\item Other possibilities: Jeff Diamond, Colin Runciman, Gerry
  Sussman, Ross King, Barak Pearlmutter, Ketil Malde, David Roundy,
  John Ioannides

\item Exclude: can't think of any. 

\end{itemize}

\noindent Yours sincerely

\noindent Tom Matheson

\end{document}