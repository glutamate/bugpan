\documentclass[11pt]{article}
\usepackage[a4paper, top=2.5cm, bottom=2.5cm, left=2.5cm, right=2.5cm]{geometry}
\begin{document}
\begin{flushright}
Tom Matheson

Department of Biology

University of Leicester

University Road

Leicester LE1 7RH

UK
\end{flushright}
\vskip2ex
Dear Editors,
\vskip2ex
\noindent We would be grateful if you would consider our attached
paper titled ``A formal mathematical framework for physiological
observations, experiments and analyses'' for publication in
\emph{Nature Methods}. Our work directly addresses issues that have
led to concerns about the reproducibility and correctness of published
research findings (e.g. Ioannidis, J (2005); PLoS Medicine 2(8):
e124). We present a new mathematical framework for physiological
experimentation that uses concepts from programming language theory
and type theory. We show that this approach provides solutions to
several outstanding problems in neuroinformatics:

\begin{itemize}

\item \emph{Ontology}: the organisation of data sharing requires a
  structure for representing a wide range of observations and
  contextual information. We propose a simple but flexible
  categorisation of heterogeneous physiological evidence, phrased in
  terms that are directly relevant to physiology.

\item \emph{Experiment definition}: we show that experiments defined
  in mathematical equations can be manipulated algebraically and
  composed from smaller components, unlike those defined in
  conventional programming languages. Our definitions are machine
  executable, so they are practically useful and unambiguous.

\item \emph{Data provenance}: before using data to test specific
  hypotheses, it is important to understand their origin. In our
  framework, \emph{any} observation can be defined in a \emph{single}
  equation that includes post-acquisition processing and censure.

\end{itemize}

We have validated our framework by using it to carry out two
non-trivial classes of experiments in neurophysiology. In one, we
measure and analyse visually evoked \emph{in vivo} spike trains, and
in a second we implement a dynamic clamp experiment to analyse
input-output relationships of vertebrate spinal neurons. Our work has
widespread implications for the structure of data sharing and
publishing: our definitions \emph{cannot} be translated to the
ontology languages used to define the semantic web. 

Our work will be of interest to experimentalists who design, implement
and analyse complex experiments, and to theoreticians concerned with
problems of data verification and novel approaches to data sharing. 

We suggest the following reviewers:

\begin{tabular}{p{6cm}  p{6cm}}
\begin{flushleft}
Marc de Kamps

School of Computing

University of Leeds

Leeds LS2 9JT

United Kingdom

dekamps@comp.leeds.ac.uk
\end{flushleft}
 &
\begin{flushleft}
Christoph Koch

Division of Biology, 216-76

Caltech

Pasadena, CA 91125

koch@klab.caltech.edu
\end{flushleft}
 \\
\begin{flushleft}
Barak Pearlmutter

Brain and Computation Lab

Hamilton Institute

National University of Ireland, Maynooth

Co. Kildare

Ireland

barak@cs.nuim.ie
\end{flushleft}
 &
\begin{flushleft}
Jerzy Karczmarczuk

Dept. d'Informatique

Av. Maréchal Juin

Campus II, S3 

14032 Caen

France

jerzy.karczmarczuk@info.unicaen.fr
\end{flushleft}
\\
\end{tabular}

\noindent Yours sincerely

\vskip5ex

\noindent Tom Matheson

\end{document}